\documentclass{article}

% Language setting
% Replace `english' with e.g. `spanish' to change the document language
\usepackage[spanish]{babel}

% Set page size and margins
% Replace `letterpaper' with`a4paper' for UK/EU standard size
\usepackage[a4paper,top=1cm,bottom=2cm,left=3cm,right=3cm,marginparwidth=1.75cm]{geometry}

% Useful packages
\usepackage{amsmath}
\usepackage{graphicx}
\usepackage[colorlinks=true, allcolors=blue]{hyperref}
\usepackage{listings}
\usepackage{amssymb}

\title{Ejercicio 2 Bloque III: Técnicas de Acceso Múltiple}
\author{André Yermak y Álvaro Hernández}

\begin{document}
\maketitle


\section{Introducción y planteamiento (??)}


\section{Resolución del problema}
\subsection{Caudal medio en j}
Sabemos que:

$$S_j = P_j \cdot P[\text{tx bien}]$$

Por lo que podemos escribir:

$$S_j = G_j \cdot \prod_{i=1}^{\substack{M}} 1 - G_i$$

Se modela la tx por Bernoulli:

$$t(x) = P^x \cdot (1-p)^{1-x} \text{ , } x \in \left\{0, 1\right\}$$


para cada tx que i no intente tx:

$$P[\text{estación "i" no intenta tx}] = 1 - G_{i}$$

$$P[ \varPhi \text{ LLegadas en P } ] = \prod_{i=1}^{\substack{M}} 1 - G_i $$




\subsection{Caudal medio en S}

Sustituiremos $G_{i}$ en la expresión calculada en el apartado anterior

$$S = G \cdot \prod_{i=1}^{\substack{M}} 1 - \frac{G}{N} \rightarrow S = G(1 - \frac{G}{M})^{M-1}$$


\subsection{Caudal medio cuando las estaciones tienden a infinito}

$$S = G \cdot \lim_{M \to \infty}(1 - \frac{G}{M})^{M-1} = Ge^{-G}$$

\subsection{Expresión para los apartados b) y c) }

$$\left| \frac{S_{\infty} - S_M}{S_{\infty}} \right| < 0.01 \rightarrow \left| S_{\infty} - S_M \right| < 0.01 \cdot \left| S_{\infty}\right|$$

Sabemos que 

$$ S_{\infty} = Ge^{-G} \quad \text{ , } \quad S_M = \frac{G}{M}e^{\frac{-G}{M}} $$

Sustituimos en la expresión planteada

$$\left| Ge^{-G} - \frac{G}{M}e^{\frac{-G}{M}} \right| < 0.01 \cdot \left| Ge^{-G}\right|$$

Ahora, sacaremos factor común de la G y se nos va operando. La que está en los exponentes, la sustituiremos por 1 ya que no afecta a los cálculos

$$\left| e^{-G} - \frac{1}{M}e^{\frac{-G}{M}} \right| < 0.01 \cdot \left| e^{-G}\right| \quad \rightarrow \quad \left| e^{-1} - \frac{1}{M}e^{\frac{-1}{M}} \right| < 0.01 \cdot \left| e^{-1}\right| $$

Y como M tiende a infinito, tomaremos $e^{\frac{-1}{M}} = 1$

$$\left| e^{-1} - \frac{1}{M} \right| < 0.01 \cdot \left| e^{-1}\right| \rightarrow M > 2.69$$

Como $M \in \mathbb{Z}$, tiene que ser 3.

\subsection{Valor de aproximación para población infinita}
\end{document}