\documentclass{article}

% Language setting
% Replace `english' with e.g. `spanish' to change the document language
\usepackage[spanish]{babel}

% Set page size and margins
% Replace `letterpaper' with `a4paper' for UK/EU standard size
\usepackage[a4paper,top=1cm,bottom=2cm,left=3cm,right=3cm,marginparwidth=1.75cm]{geometry}

% Useful packages
\usepackage{amsmath}
\usepackage{graphicx}
\usepackage[colorlinks=true, allcolors=blue]{hyperref}
\usepackage{fancybox}

\title{Ejercicios Mates II}
\author{Alvaro Hernandez}
\date{04 de Febrero de 2024}

\begin{document}
\maketitle


%\begin{abstract}
%Your abstract.
%\end{abstract}


\section{Ejercicios}


\boxed{1} Dados dos planos:
\begin{gather*}
    \pi_1 :  4x + 6y - 12z + 1 = 0 \quad  \quad \pi_2 : -2x - 3y + 6z - 5 = 0
\end{gather*}

\begin{enumerate}
    \setcounter{enumi}{0} % Establece el contador de la lista en 0
    \item[a)] Halla el volumen que formaría un paralelepípedo que tiene como dos de sus caras $\pi_1$ y $\pi_2$.
    \item[b)] ¿Y si en vez de un paralelepípedo fuera un cubo?
\end{enumerate}

\quad
\quad
\quad

\boxed{2} Considere la recta y los planos:
\begin{gather*}
    r :  \frac{x-2}{-1} = \frac{y-2}{3} = \frac{z-1}{1}\quad  \quad \pi_1 : x = 0 \quad  \quad \pi_2 : y = 0
\end{gather*}
\begin{enumerate}
    \setcounter{enumi}{0} % Establece el contador de la lista en 0
    \item[a)] Halla los puntos de la recta r que equidistan de los planos $\pi_1$ y $\pi_2$.
    \item[b)] Determina la posición relativa de la recta r y la recta intersección de los planos $\pi_1$ y $\pi_2$.
\end{enumerate}

\end{document}