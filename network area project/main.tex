\documentclass{article}

% Language setting
% Replace `english' with e.g. `spanish' to change the document language
\usepackage[spanish]{babel}

% Set page size and margins
% Replace `letterpaper' with `a4paper' for UK/EU standard size
\usepackage[a4paper,top=2.5cm,bottom=2.5cm,left=3.5cm,right=3.5cm,marginparwidth=2.5cm]{geometry}

% Useful packages
\usepackage{amsmath}
\usepackage{graphicx}
%Podemos cambiar el color del Índice(negro), los enlaces están marcados en azul. 
\usepackage[colorlinks=true, linkcolor=black, urlcolor=blue]{hyperref}
\usepackage{fancybox}
\usepackage{listings}
\usepackage{subcaption}
%\lstset{
%    language=Matlab,
%    extendedchars=true
%}

\title{Diseño de Redes Inalámbricas en La Costera}
\author{Andé Yermak y Álvaro Hernández}
%\date{}

\begin{document}
\maketitle

%Aquí se generará nuestro Índice, podemos poner /newpage para tenerlo solo en la página
\tableofcontents
\newpage

\section{Introducción}

Se pretende dotar de acceso a internet a determinadas zonas de La Costera, Murcia. Una pedanía situada cerca de Alhama donde residen alrededor de 400 habitantes. La idea planteada será personalizar, según la necesidad, cada una de las zonas propuestas para su funcionamiento, y a su vez reducir al máximo el coste de instalación.

\quad

Al ser un servicio propuesto por el ayuntamiento, se usarán bandas de frecuencia de la red troncal, es decir, de $5GHz$ y de $2.4GHz$, donde no será necesario cierta cantidad de trámites para usarlas a comparación de otras personalizadas, además de ser gratis el añadido de radioenlaces. Por otro lado, se limitarán ciertos parámetros como la potencia de transmisión.

\subsection{Planteamiento de TeleClub}

El teleclub es un centro social que necesita de internet tanto \textbf{indoor} como \textbf{outdoor}. De puertas para dentro estará la biblioteca, lugar donde la conexión es de suma importancia para tareas como descarga de ebooks o registros en su web de usuarios. En el exterior del teleclub, dotaremos de internet básico para el ocio que puedan tener los visitantes y residentes, se tiene en cuenta que el exterior tendrá un mayor número de usuarios.

(ver num de usuarios)

\subsection{Planteamiento de la Nave}

(30 usuarios con 4 puntos de acceso y los servidorés necesitarán hasta $10Mbps$)

\subsection{Planteamiento de Paneles}

(Se suelen robar paneles por lo que necesitaremos añadir videovigilancia. No hace falta añadir acceso a internet para usuarios)

\section{Materiales elegidos}

Se ha proporcionado en el aula virtual un \href{http://www.microcom.com.ar/fotos/ficha7097LBE-M5-23.compressed.pdf}{Datasheet} donde aprovecharemos para elegir el material que más se adapte a nuestros planteamientos. 
\end{document}