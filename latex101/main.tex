% Para trabajos como TFG usar documentclass book
\documentclass[12pt]{article}

% Language setting
\usepackage[spanish]{babel}

% Set page size and margins
\usepackage[a4paper,top=2cm,bottom=2cm,left=3cm,right=3cm,marginparwidth=1.75cm]{geometry}

% Useful packages
\usepackage{amsmath}
\usepackage{graphicx}
\usepackage[colorlinks=true, linkcolor=black, urlcolor=blue]{hyperref}
\usepackage{color}

% Uncomment these packages if you want to use dark mode
% \usepackage{darkmode}
% \enabledarkmode

% Title and author
\title{\LaTeX 101}
\author{Álvaro Hernández}
\date{\today}

\begin{document}

%%%%%%%%%%%%%%%%%%%%%%%%%%%%%%%%%%

\maketitle
\tableofcontents
\newpage

\section{Sección 1}
Hola mundo

\subsection{Prueba2}
Tipografías distintas:

\texttt{emph}: \emph{enfática}

\texttt{textbf}: \textbf{negrita}

\texttt{texttt}: \texttt{máquina}

\texttt{underline}: \underline{subrayado}

\texttt{textnormal}: \textnormal{normal}\ldots

\subsection{Colores}

\textbf{importante: } cargar primero el paquete color con \verb|\usepackage{color}|

Se pueden usar colores como {\color{red}{ rojo}}, {\color{blue}{azul}} o {\color{green}{ verde}}

Las llaves de {rojo} o {verde} pueden sobrar si hemos puesto las llaves del exterior o {verde} pueden sobrar si hemos puesto las llaves del exteriorr

\subsection{Tamaños}

\texttt{tiny}: {\tiny{minusculo}}

\texttt{small}: {\small{pequeño}}

\texttt{normalsize}: {\normalsize{normal}}

\texttt{large}: {\large{grande}}

\texttt{Large}: {\Large{grande}}

\texttt{Huge}: {\Huge{enorme}}

\subsection{Caracteres especiales}

\subsubsection{Guiones}

Tres posibles longitudes de guiones: corto (2024-2025), medio (2024--2025) y largo (2024---2025)

\subsubsection{Comillas}

Para las comillas haremos con la derecha de la p:

``Apertura de comillas y cierre con''


\subsubsection{Caracteres reservados}

Caracteres reservados:

Antes de cada caracter de este tipo lo hare primero con la barra invertida pero si quiero poner esa barra invertida en el texto usaré \verb|\textbackslash|

(Usar texttt con las demas...)
\texttt{\$}
\&
\%
\#

Para poner llaves \{ \}


\end{document}


